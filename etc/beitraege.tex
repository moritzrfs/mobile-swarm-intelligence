% LTeX: language=de-DE
\section*{Beitrag der Autoren}
Die nachfolgende Tabelle gibt einen Überblick über die Aufgabenverteilung innerhalb der Studienarbeit \textit{\thetitle}, die von drei Autoren verfasst wurde. Jeder Autor übernahm spezifische Aufgaben und trug zur Fertigstellung der Arbeit bei. Die Tabelle stellt transparent dar, welche Person für welche Teile der Arbeit verantwortlich war und verdeutlicht die individuellen Beiträge jedes Autors zur Erstellung dieser Arbeit.
\vspace*{1cm}

\begin{table}[H]
    \renewcommand{\arraystretch}{1.2}
    \begin{tabularx}{\textwidth}{|X|X|}
        \hline
        \textbf{Abschnitt} & \textbf{Autor} \\
        \hline
        Abstract & \textit{Jonas Grohe, Steven Hartinger,\newline Moritz Reufsteck} \\
        \hline
        \autoref{ch:einleitung}: \autoref{sec:problemstellung} & \textit{Steven Hartinger} \\
        \hline
        \autoref{ch:einleitung}: \autoref{sec:ziel} & \textit{Moritz Reufsteck} \\
        \hline
        \autoref{ch:einleitung}: \autoref{sec:vorgehensweise} & \textit{Steven Hartinger} \\
        \hline
        \autoref{ch:schwarmintelligenz}: \autoref{sec:natur} & \textit{Steven Hartinger} \\
        \hline 
        \autoref{ch:schwarmintelligenz}: \autoref{sec:mobile} & \textit{Moritz Reufsteck} \\
        \hline
        \autoref{ch:algorithmen} & \textit{Jonas Grohe} \\
        \hline
        \autoref{ch:comparison} & \textit{Jonas Grohe} \\
        \hline
        \autoref{ch:anwendungAlgo} & \textit{Steven Hartinger} \\
        \hline
        \autoref{ch:pi} & \textit{Moritz Reufsteck} \\
        \hline
        \autoref{ch:anwendungRobot} & \textit{Steven Hartinger} \\
        \hline
        \autoref{ch:fazit} & \textit{Jonas Grohe, Steven Hartinger,\newline Moritz Reufsteck} \\
        \hline
    \end{tabularx}
\end{table}