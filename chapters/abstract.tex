% LTeX: language=de-DE
\begin{abstract}
Im Rahmen dieser Arbeit wird ein Roboter entwickelt, der in der Lage ist, eigenständig eine Strecke mithilfe eines Schwarmintelligenz-Algorithmus zu bewältigen.
Für das Verständnis der Arbeit wird zunächst die Theorie der Schwarmintelligenz erläutert. Hierfür werden die Algorithmen Particle Swarm Optimization, Ant Colony Optimization, und Artificial Bee Colony erläutert sowie deren Stärken und Schwächen aufgezeigt. 
Anhand geeigneter Kriterien wird ein Artifical Bee Colony Algorithmus ausgewählt, der zur Lösung des Problems der Wegfindung auf einer Karte mit Hindernissen verwendet wird.
Die Implementierung ist auf das Problem zur Wegfindung auf einer Karte mit Hindernissen angepasst. Dabei wird die stetige Gleichverteilungsfunktion verwendet, um zufällige Punkte innerhalb der Karte zu generieren, die dann auf ihre Gültigkeit überprüft werden. Die Gültigkeit eines Punktes wird durch zwei Bedingungen definiert: Der Punkt darf nicht innerhalb eines Hindernisses liegen und eine Verbindung von zwei Punkten darf kein Hindernis schneiden. Die Bewertung eines Pfades wird durch die Summe der euklidischen Abstände zwischen den Punkten berechnet. Die lokale Suche des Bienenalgorithmus wurde dabei genutzt, um die Punkte in einer bestimmten Nachbarschaft gegebenenfalls zu optimieren. Eine Ausnahme bildet der letzte Punkt des Pfades, der unter Umständen keinen gültigen Weg zum Ziel haben kann und daher durch einen neuen Punkt ersetzt wird. Das Ergebnis des Algorithmus liefert den besten Pfad der gefunden werden konnte. 
Für die Umsetzung des Vorhabens für den Roboter werden drei Konzepte erläutert, miteinander verglichen und anschließend das passendste Konzept anhand von Kriterien ausgewählt. Dabei werden Faktoren wie die verfügbare Hardware und vorhandene Kenntnisse berücksichtigt. Besonderes Augenmerk wird auf die Kommunikation zwischen den einzelnen Komponenten gelegt, wobei Aspekte wie Sicherheit, Authentifizierung und Validierung von Daten einbezogen wird. Hierbei wird eine Ansteuerung mittels einer REST API gewählt. Damit hierdurch übertragene Daten korrekt verarbeitet werden können, wird eine Funktion zur Übersetzung von Instruktionen für die Anwendung auf dem Einplatinencomputer implementiert und integriert. Schließlich wird die Ergebnisse der Planung mit den Hardware-Komponenten vereint, eine geeignete Plattform für den Roboter konstruiert und alles in einem Prototyp zusammengeführt.
\end{abstract}