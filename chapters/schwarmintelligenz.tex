% LTeX: language=de-DE
\section{Ansätze aus der Natur}
In dieser Arbeit werden grundlegend drei verschiedene Algorithmen der Schwarmintelligenz für die vorliegende Problematik erläutert und evaluiert. Jeder dieser Algorithmen ist eine Analogie zu dem Verhalten von Lebewesen aus der Natur. Schwarmoptimierungsalgorithmen (SOAs) imitieren die kollektive Explorationsstrategie von Schwärmen in der Natur bei Optimierungsproblemen. Diese Algorithmen verwenden einen populationsbasierten Ansatz für die Probleme. Diese Gruppe von Algorithmen wird als populationsbasierte stochastische Algorithmen bezeichnet \cite{Yuce2013}.

Der erste Algorithmus ist der Bees Algorithm (BA). Dieser basiert auf der Nahrungssuche von Honigbienen. Honigbienen sammeln ihr Essen über mehrere Kilometer, weshalb ein organisierter Ablauf und Kommunikation die Schlüssel des Erfolges sind. Die Bienen gehen dabei so vor, dass sie sogenannte Späher-Bienen zur Nahrungsfindung aussenden. Sind diese Späher erfolgreich, kehren sie zurück zum Bienenstock und informieren die anderen Bienen per Tanz über den Standort des Essens. Anhand dieser Information können die anderen Honigbienen das Essen sammeln, während die Späher versuchen, weitere Blumenbeete ausfindig zu machen \cite{Brownlee2011}.

Ein weiterer relevanter Algorithmus zur Optimierung eines Schwarms ist die Particle Swarm Optimization (PSO). Dieser nutzt das Verhalten von Populationsgruppen wie beispielsweise Herdentieren oder  Vogel- bzw. Fischschwärmen. Dabei steht der Schwarm für die Population und jedes Mitglied der Population wird Partikel genannt. Die Partikel sind bei der Optimierung entscheidend, da sie das globale Optimum finden sollen.  Dabei wird ein Partikel abhängig von seiner Position und Geschwindigkeit zu der besten bekannten Position geführt und informiert die anderen Partikel über seine Position. Falls nun ein anderer Partikel eine bessere Position als die vorherige gefunden hat, werden die anderen Partikel benachrichtigt und der Schwarm in die Richtung gelenkt. Dies geschieht so lange, bis der Schwarm innerhalb des Suchraumes die optimale Position gefunden hat \cite{Brownlee2011}. 

Der letzte Algorithmus, der in dieser Arbeit vorgestellt wird, ist der Ant System Algorithmus. Inspiriert ist dieser Algorithmus von der Essenssuche von Ameisen. Ameisen sind so gut wie blind weshalb sie bei der Nahrungsfindung über Pheromone kommunizieren. Schüttet eine Ameise Pheromone aus, bedeutet es, dass sie Nahrung gefunden hat. Dies geschieht jedes Mal, sobald eine Ameise Essen findet. Durch das positive Feedback wissen die Ameisen also, welcher Route sie folgen müssen. Über die Zeit verfallen die Pheromone, damit alte Pfade nicht mehr gefolgt werden \cite{Brownlee2011}

\section{Mobile Schwarm Intelligenz}
Durch komplexer werdende Probleme in allen Bereichen des Lebens ist es von Relevanz, dass schnelle und effiziente Ansätze zur Lösung von Problemen gefunden werden. Besonders wenn ein Informationsaustausch oder lokale Kommunikation einer Mehrzahl von Systemen stattfindet, kann Schwarm Intelligenz (oder auch Kollektive Intelligenz genannt) bei der Lösung von Problemstellungen helfen. Als Inspiration vieler Grundlagen Schwarm-basierter Systeme dient das biologische Verhalten von Vorkommnissen in der Natur, wie das von Insekten wie beispielsweise Ameisen, oder das Verhalten von in Schwärmen agierenden Lebewesen wie Fischen oder Vögeln \cite{Blum2008}. 
Ein Schwarm, der in der Natur als kollektives Verhalten von organisierten, jedoch dezentralisierten Populationen bekannt ist \cite{Kiranyaz2013}, wird im Jahr 1999 mit der Informationstechnik von \cite{Bonabeau1999} in Zusammenhang gebracht und als Eigenschaft bezeichnet, die es individuellen Komponenten durch Interaktion ermöglicht, Aktionen durchzuführen, die nicht durch das alleinige Handeln einer einzeln agierenden Komponente umsetzbar sind. Interaktionen, wie beispielsweise das Abgeben von Pheromonspuren auf gefundenen Wegen zwischen dem Nest einer Kolonie von Ameisen, oder dem sogenannten "waggle dance" von Bienen, deren Schwärmer diesen praktizieren, wenn sie auf neue Futterquellen gestoßen sind \cite{Blum2008, Panigrahi2011}, können in Form anderer Kommunikations- oder Interaktionsverfahren im Netzwerk- oder Datenverkehr angewendet werden. Ersteres basiert darauf, dass einzelne Ameisen das abgegebene Pheromon anderer Ameisen als indirektes Kommunikationsmedium nutzen, um damit den kürzesten Weg zu einer Futterquelle zu finden \cite{Blum2008}. Die Ameisen nutzen dabei die Dichte des Pheromons, um die Richtung zu bestimmen, in die sie sich bewegen sollen. Um dieses Verhalten auf Anwendung in der Informationstechnik übertragen zu können, werden künstliche Pheromon Spuren geschaffen, deren Informationsgehalt dadurch genutzt werden kann, um probabilistische Entscheidungen zu treffen. Abgegeben werden diese künstlichen Spuren dann von Agenten, die die künstlichen Ameisen darstellen \cite{Gendreau2010}.

Ein bekannter Algorithmus, der in vielen wissenschaftlichen Arbeiten wie \cite{Blum2008,Parpinelli2011,Brownlee2011} thematisiert und erläutert wird, ist die Ant-Colony-Optimization (ACO). Hierbei handelt es sich um eine metaheuristische Technik im Verhalten von Ameisen. Das Ziel der Spezies ist es, den kürzesten Weg einer Kolonie bis zu einer Futterquelle zu ermitteln \cite{Parpinelli2011}. Der aus der Biologie bekannte und definierte Schwarm wird im Jahr 1999 von \cite{Bonabeau1999} als Eigenschaft bezeichnet, die es individuellen Komponenten durch Interaktion ermöglicht, Aktionen durchzuführen, die nicht durch das alleinige Handeln einer einzeln agierenden Komponente umsetzbar sind. 
Die in der Natur vorkommenden Phänomene von Schwarmintelligenz haben zudem Forscher dazu motiviert, \textit{intelligent mutli-agent} Systeme zu entwerfen, deren Verhalten sich stark an den Vorkommnissen und Vorbildern in der Natur orientiert. Besonders Optimierungsprobleme, beispielsweise in den Bereichen des Routings von Telekommunikationsdaten, profitieren durch die Implementierung von Schwarmintelligenz Algorithmen \cite{Blum2008}. Zudem ist das Themengebiet der Robotik ebenfalls ein Anwendungsfall, für den das ursprünglich in der Natur vorkommende Phänomen als Vorbild dient.
Im Folgenden werden drei bekannte Anwendungsfelder, die als verschiedene Anwendungsfälle für Schwarmintelligenz in einer Vielzahl von Literaturwerken \cite{Blum2008,Parpinelli2011,Eberhart2001,Spears2005} erläutert werden.
Das Phänomen der Ameisen, aus dem die metaheuristische Technik ACO (Ant Colony Optimization) hervorgeht \cite{Blum2008, Panigrahi2011, Gendreau2010}, stellt eine effektive Methode dar, um Routing Probleme in Netzwerken zu lösen. Eine weit verbreitete und umfangreich erforschte Technik hierfür ist AntNet. AntNet basiert darauf, dass Agenten zu gleicher Zeit die Gegebenheiten, das heißt, Knoten und Kanten in einem Netzwerk erforschen und währenddessen die gesammelten Informationen austauschen. Wie im biologischen Vorbild findet dieser Informationsaustausch ebenfalls indirekt statt \cite{DiCaro2011}. 
Eine weitere relevante Optimierungsmethode von Problemlösungen ist die PSO (Particle Swarm Optimization), die zum ersten Mal 1995 in \cite{Kennedy1995} thematisiert wurde und ihren Ursprung wie der ACO ebenfalls an einem der Biologie zugehörigen Phänomen hatte. Hierbei handelt es sich um einen probabilistischen Algorithmus, bei dem individuelle Partikel als Teil eines Schwarms in einem Suchraum vorhanden sind. Jedes einzelne Partikel stellt eine potenzielle Lösung für das zu behandelnde Problem dar. Diese Partikel bewegen sich durchgehend durch den Suchraum. Das Ziel hierbei ist, dass eine optimale oder ausreichend gute Lösung gefunden wird. Während des gesamten Vorgangs und weiterführenden Iterationen beobachten Partikel jeweils die Positionen der anderen Partikel. Nach jeder Iteration speichert ein Partikel den besten Fitnesswert, der während einer Suche gefunden wurde  \cite{Blum2008}.
Ein weiterhin relevanter Algorithmus, der unter anderem in den Bereichen Robotik und Netzwerk Routing Anwendung findet, ist der Bees Algorithm (BA) Einer der Algorithmen basiert auf der dezentralisierten Entscheidungsfindung von Bienenschwärmen bei der Nahrungssuche. Bienen praktizieren hierbei einen Tanz, passen ihn je nach Profitabilität der Futterquelle an und locken damit andere Bienen an. Müssen Bienen zwischen zwei Kandidaten entscheiden, wählen sie diejenige, deren waggle-dance stärker ausgeprägt ist \cite{Panigrahi2011, Yuce2013, Parpinelli2011}. Ist die Futterquelle einer anderen Biene vielversprechender als die, zu der sich eine Biene zu diesem Zeitpunkt bewegt, verwirft sie die eigene und macht sich auf den Weg zu der Quelle, die die andere Biene preisgegeben hat \cite{Bonabeau1999}. Die Analyse des resultierenden Algorithmus zeigt die Existenz von zwei Typen von Bienen: Scout-Bienen und Arbeiter-Bienen. Die Arbeiter-Bienen nutzen die bereits von den Scout-Bienen gesammelten Informationen, um eine erfolgreiche Lokalisierung von Nahrungsquellen zu erreichen. Die Scout-Bienen hingegen erkunden die Umgebung stochastisch und teilen ihre gewonnenen Lösungen mit der Kolonie \cite{Parpinelli2011}. 
Ant-Colony-Optimization (ACO), Bee-Algorithm (BA) und Particle-Swarm-Optimization (PSO) sind drei wichtige Algorithmen für Schwarmintelligenz. Sie nutzen Konzepte aus dem Verhalten von Tieren in Schwärmen oder Kolonien, um Lösungen für komplexe Probleme zu finden. ACO nutzt das Verhalten von Ameisen, um komplexe Pfade zu finden. BA nutzt das Verhalten von Bienen, um optimale Lösungen zu identifizieren. PSO nutzt Partikel-basierte Bewegungen, um optimale Lösungen zu erreichen. Sie stellen Ansätze bereit, um Probleme mithilfe eines kollektiven Verhaltens von Agenten, in diesen Fällen Ameisen, Bienen und Partikel, zu lösen und ermöglichen die Modellierung von komplexen Systemen. 
Zusammenfassend lässt sich sagen, dass die Erläuterung der relevanten Algorithmen für Schwarmintelligenz ein wichtiger Schritt ist, um die Konzepte mobiler Schwarmintelligenz zu verstehen. Mobiler Schwarmintelligenz beschreibt die Anwendung von Prinzipien der Schwarmintelligenz in einem mobilen Kontext, bei dem eine Gruppe von selbständigen Geräten oder Agenten zusammenarbeitet, um gemeinsame Ziele zu erreichen. Gängigerweise sind die mobilen Systeme in der Lage, sich innerhalb einer physischen Umgebung fortzubewegen und diese zu erkunden. Die Algorithmen ACO, BA und PSO haben somit hohe Relevanz für die Anwendung in mobilem Schwarmverhalten aufgrund ihrer Ursprünge in der Natur zur Problemlösung mithilfe von kollektiven Verhaltens- und Zusammenarbeitsstrategien für Modellierung und Optimierung. Es ist jedoch zu berücksichtigen, dass die Wahl des geeigneten Algorithmus abhängig von den spezifischen Anforderungen und Einschränkungen des zu lösenden Problems ist. Es kann vorkommen, dass andere Algorithmen in bestimmten Kontexten besser geeignet sind. 
