% LTeX: language=de-DE
\label{ch:fazit}
\section{Ausblick}
Das Feld der Metaheuristik ist ein Feld, welches sich schnell verändert und wächst. Es werden immer wieder neue Verbesserungen zu bestehenden oder komplett neue Algorithmen veröffentlicht. Außerdem wurden hier nur drei Algorithmen näher betrachtet. Um eine bestmögliche Performance sicherzustellen, müssten mehr Algorithmen sowie dazugehörige Verbesserungen betrachtet und bestenfalls getestet werden. Zusätzlich müsste dies immer wieder aktualisiert werden, sobald eine mögliche Verbesserung vorliegt.
Der beschriebene Bienenalgorithmus zur Wegfindung auf einer Karte mit Hindernissen hat vielversprechende Ergebnisse gezeigt und könnte in der Zukunft für die Navigation von autonomen Robotern oder Drohnen eingesetzt werden. Weiterführende Forschung könnte sich darauf konzentrieren, den Algorithmus für dynamische Umgebungen zu optimieren und seine Anwendbarkeit auf größere Karten zu erweitern. Zudem könnten verschiedene Bewertungsfunktionen für Pfade getestet werden, um die Effizienz des Algorithmus zu verbessern. Insgesamt hat die Arbeit gezeigt, dass der Bienenalgorithmus eine vielversprechende Methode für die Wegfindung in schwierigen Umgebungen darstellt.

Für die Umsetzung des Roboters wurden drei mögliche Konzepte miteinander verglichen. Die Wahl für die Implementierung und Umsetzung des Roboters fiel auf das Konzept mit einer hybriden Umsetzung, wobei ein Client einen Schwarmintelligenz-Algorithmus ausführt, daraus Instruktionen für den Roboter generiert und diese an den Roboter sendet. Bei dieser Form der Implementierung werden jedoch keine Umgebungsdaten aktiv durch den Roboter gesammelt. Um dieses Vorhaben zu realisieren, hätte der Roboter weiterhin mit Sensoren ausgestattet werden müssen, welche die Umgebung erfassen. Diese Daten müssten dann in eine Form gebracht werden, mit der der verwendete Schwarmintelligenz-Algorithmus umgehen kann. Dies hätte die Implementierung im Rahmen dieser studentischen Arbeit deutlich erschwert. Grund dafür war der bereits hohe Aufwand bei der Ansteuerung der verwendeten Motoren. Besonders die Kontrolle der Geschwindigkeiten der Motoren war nicht zu lösen. Die Implementierung des Roboters mit der hybriden Umsetzung hat gezeigt, dass die Umsetzung eines Roboters mit Schwarmintelligenz-Algorithmus möglich ist. Zukünftig wäre eine umfangreichere Umsetzung unter der Verwendung von Sensoren denkbar, um die Umgebung des Roboters zu erfassen und eine Umsetzung dieser Variante zu ermöglichen. Hierbei würden die durch die Sensoren erfassten Umgebungsdaten durch den Schwarmintelligenz-Algorithmus verarbeitet werden, wodurch eine gesteigerte Realitätstauglichkeit des Roboters erreicht werden könnte.