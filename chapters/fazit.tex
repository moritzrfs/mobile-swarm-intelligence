% LTeX: language=de-DE

\section{Ausblick}
Das Feld der Metaheuristik ist ein Feld, welches sich schnell verändert und wächst. Es werden immer wieder neue Verbesserungen zu bestehenden oder kommplett neue Algorithmen veröffentlicht. Außerdem wurden hier nur drei Algorithmen näher betrachtet. Um eine bestmögliche Performance sicherzustellen, müssten mehr Algorithmen sowie dazugehörige Verbesserungen betrachtet und bestenfalls getestet werden. Zusätzlich müsste dies immer wieder aktualisiert werden, sobald eine mögliche Verbesserung vorliegt.
Der beschriebene Bienenalgorithmus zur Wegfindung auf einer Karte mit Hindernissen hat vielversprechende Ergebnisse gezeigt und könnte in der Zukunft für die Navigation von autonomen Robotern oder Drohnen eingesetzt werden. Weiterführende Forschung könnte sich darauf konzentrieren, den Algorithmus für dynamische Umgebungen zu optimieren und seine Anwendbarkeit auf größere Karten zu erweitern. Zudem könnten verschiedene Bewertungsfunktionen für Pfade getestet werden, um die Effizienz des Algorithmus zu verbessern. Insgesamt hat die Arbeit gezeigt, dass der Bienenalgorithmus eine vielversprechende Methode für die Wegfindung in schwierigen Umgebungen darstellt.