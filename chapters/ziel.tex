% LTeX: language=de-DE
\section{Ziel der Arbeit}
Schwarmintelligenz beschreibt das Phänomen, dass ein Schwarm von Lebewesen oder künstlichen Agenten in der Lage ist, gemeinsam komplexe Aufgaben zu lösen, die für einzelne Individuen unlösbar wären. Schwarmintelligenz findet sich in der Natur bei vielen Arten, wie beispielsweise Ameisen, Bienen oder Schwärmen von Vögeln oder Fischen. Der Schlüssel zur Schwarmintelligenz liegt in der Fähigkeit der Individuen, miteinander zu kommunizieren und Informationen auszutauschen.
In der Technologie wird Schwarmintelligenz zunehmend genutzt, um komplexe Aufgaben zu lösen. Mobile Schwarmintelligenz beschreibt dabei den Einsatz von Schwarmintelligenz in mobilen Systemen wie beispielsweise Robotern oder Drohnen. Diese können sich eigenständig in einer Umgebung bewegen und gemeinsam Aufgaben erledigen, die für einzelne Roboter oder Drohnen zu schwierig wären.
Um Schwarmintelligenz in mobilen Systemen zu nutzen, gibt es eine Vielzahl von Algorithmen. Beispiele hierfür sind der Schwarm-Algorithmus, der Ameisenalgorithmus oder der Partikel-Schwarm-Algorithmus. Diese Algorithmen nutzen unterschiedliche Strategien, um Schwarmintelligenz in mobilen Systemen zu erzeugen.
Im Rahmen dieser Arbeit soll ein geeigneter Algorithmus für die Navigation eines Einplatinencomputers durch einen Hindernisparcours mittels Schwarmintelligenz ermittelt werden. Hierfür sollen verschiedene Algorithmen vorgestellt und miteinander verglichen werden. Der ausgewählte Algorithmus soll es dem Einplatinencomputer ermöglichen, Hindernisse eines Parcours zu verarbeiten und eine optimale Route durch den Parcours zu finden.
Damit die Schwarmintelligenz-Algorithmen auf dem Einplatinencomputer ausgeführt werden können, ist eine zuverlässige Kommunikation mit einem Server notwendig. Hierfür können beispielsweise WLAN- oder Bluetooth-Verbindungen genutzt werden. Die Kommunikation muss dabei praktisch und zuverlässig umgesetzt werden, um eine fehlerfreie Navigation durch den Hindernisparcours zu gewährleisten.
Um den Einplatinencomputer für die Navigation durch den Parcours fit zu machen, muss dieser zu einem Roboter umgebaut werden. Hierfür werden geeignete Bauteile benötigt, die dem Roboter die Fähigkeit zum Fahren, Lenken und Orientieren geben. Die Bewegung des Roboters kann mithilfe von Servomotoren umgesetzt werden, die eine präzise Steuerung der Bewegungen ermöglichen.
Insgesamt zeigt sich, dass Schwarmintelligenz ein vielversprechendes Konzept für die Navigation von mobilen Systemen in komplexen Umgebungen ist. Durch die Nutzung von Schwarmintelligenz-Algorithmen können Roboter oder Drohnen eigenständig und effizient Aufgaben erledigen, die für einzelne Individuen zu schwierig wären.